\documentclass[12pt]{report}
\usepackage[T1]{fontenc}
\usepackage[utf8]{inputenc}
\usepackage{graphicx}
\usepackage{geometry}
\usepackage{hyperref}

\setcounter{secnumdepth}{4}

\graphicspath{ {figures/} }
\geometry{ a4paper, margin=1in}

\title{
	\begin{figure}
		\includegraphics[scale=0.3]{logo_upc}
		\raggedleft
		\vspace{2.5cm}
	\end{figure}
  \textbf{ETH-Paid Internet Access} \\
  Master Project
}

\author{
  A. Egio, L. Gonz\'alez, A. Pardo, T. Romero\\
  UPC School, Carrer de Badajoz 73, 08005 Barcelona - Catalunya\\
  E-mail: [alfonsoea, lluisgv2, albertopm5, joserc25]\\
}

\date{Barcelona   \today}


\begin{document}

  \maketitle

  \tableofcontents

  \chapter*{Abstract}
  \addcontentsline{toc}{chapter}{Abstract}
  This document shows a new solution to get Internet acces to users using Wifi networks. The innovation is about to add a time per use payment gateway using the ethereum ETH crytocurrency and tokens ( new Ethereum's cryptoactives)

  \chapter*{Gratefulness}
  \addcontentsline{toc}{chapter}{Gratefulness}
  We want to give thanks to ........

  \chapter{Introduction} \label{sec:introduction}
  \textbf{ See yellowPaper}

  \section{Objective} \label{ch:objective}
  In this section we explain the first aproach to give a solution to our proposal and the change of orientation to avoid the main difficulty
	....
	\subsection{Original idea} \label{ch:original-idea}

  The original idea was :
  \textbf{Offer a technical solution for anyone with internet conexion can offer a pay Wifi conexion using blockchain and cryptocurrency as a service for a Internet connection}

  The main problem is the \textbf{legislation}. For example in Spain is no possible that a private person can resell an Internet connection.
	......
	\subsection{New Aproach} \label{ch:new-aproach}
  We changue our aproach to focus on a possible problem to solve. There are a lot of travelers that use mobile devices and some of them have cryptocurrency.   If these travelers are traveling and they want to use an Internet acces point using a prepaid system ( they pay for a temporal Internet acces). By the moment it's not possible to pay this service with cryptocurrency, because the system don't let it.

  Our proposal is give a service that let acces to Internet via Wifi using a prepaid method with crytocurrency and tokens.

  \section{Similar Projects} \label{ch:similar-projects}
  \textbf{ See yellowPaper}

  \subsection{Fon} \label{ch:fon}
  \textbf{ See yellowPaper}

  \subsection{Hotspot me!} \label{ch:ethberlinzwei}
  Presented in in ETH Berlin Zwei Hackathon.
  Links : \href{https://devpost.com/software/hotspot-me}{Devpost Web]} , \href{https://github.com/freeatnet/ethBerlin-HotspotMe}{HotspotMe Github}
	\textbf{ See yellowPaper}

	\section{What is a Hotspot?} \label{ch:hotspot}

	\subsection{Hotspot Concept} \label{ch:hotspot-concept}
  The internet access system to which we refer, is known as Hotspot. Access to this system is through a mobile device with Wi-Fi. When the user's device is connected to the Wi-Fi offered by the hotspot, it accesses through the browser a web page that offers the Hotspot. This website is known as captive portal. This captive portal allows you to select the time of use for the user to make the payment through the payment system that the Hotspot allows. These systems are: credit card, paypal, etc

	\subsection{Actual Hotspot Solutions (Year 2019)} \label{ch:hotspot-solution}

  Free or paid options (online payment by card or other form) that has a user who wants to use a hotspot. (Source: self made. They may not all be)

  Options :
  \begin{itemize}
    \item Freewifi
    \item Freewifi subject to product consumption
    \item Hostpot / router rental
    \item Hostpot / router of the user (prepaid data sim card is required)
    \item Paid or prepaid Hostspot for use
    \item Data prepaid SIM
  \end{itemize}

  \subsubsection{Freewifi} \label{ch:freeWifi}

  \paragraph*{Service Offer}

  Restaurants and hotels, Department stores, shops, shopping centers, passenger transport by subway / bus / train / plane,
  airports, subway / train / bus stations, town halls or others public buildings, schools, etc.

  \paragraph*{Advantages}

  \begin{itemize}
    \item Free
  \end{itemize}

  \paragraph*{Disadvantages}

  \begin{itemize}
    \item This places can get information for personal marketing campaings.
    \item if the place have paid option, the free option is going to have a slow connection.
    \item Limited content acces or no acces for some web services : video and music online players, TV online, etc...
  \end{itemize}

  \subsubsection{Freewifi subject to product consumption} \label{ch:freeWifi-consumtion}

  \paragraph*{Service Offer}
  Restauration sector.

  \paragraph*{Advantages}

  \begin{itemize}
    \item Free
  \end{itemize}

  \paragraph*{Disadvantages}

  \begin{itemize}
    \item Internet connection limited by time.
  \end{itemize}

  \subsubsection{Hostpot / router rental} \label{ch:router-rental}

  \paragraph*{Service Offer}
  Router rental companies. Examples:
  \begin{itemize}
    \item \href{https://alldayinternet.com/}{AllDayInternet}
    \item \href{https://www.spaininternet.net/es/alquiler-wifi-spain/}{SpainInternet}
    \item \href{https://wifivox.com/es/}{Wifibox}
  \end{itemize}

  \paragraph*{Advantages}
  \begin{itemize}
    \item Can be used in all the countries where the router provider has an agreement.
  \end{itemize}

  \paragraph*{Disadvantages}
  \begin{itemize}
    \item It is necessary to process the sending, collection and return of the router. Or go to a
    router pick-up / drop-off point.
  \end{itemize}


  \subsubsection{Hostpot / router of the user (prepaid data sim card is required)} \label{ch:router-ownership}

  \paragraph*{Service Offer}
  The user himself is the owner of the hotspot / router, and can use it anywhere with mobile coverage and SIM support.
  Products example:
  \begin{itemize}
    \item Huawei High Speed Unlocked 4G/ LTE E5573 150 Mbps.
    \item \href{https://es.support.t-mobile.com/docs/DOC-2384}{T-sysmtes Smartphone Mobile HotSpot (compartir/conectar a Wi-Fi)}
    \item \href{https://es.support.t-mobile.com/community/phones-tablets-devices#hotspot/}{T-sysmtes devices}
  \end{itemize}

  \paragraph*{Advantages}
  \begin{itemize}
    \item Can be used anywhere that accepts this type of device and has mobile telephony coverage.
  \end{itemize}

  \paragraph*{Disadvantages}
  \begin{itemize}
    \item You must have a SIM card with data for the country you wish to visit or Data SIM of an Operator with international agreements.
    \item The router / hotspot must be compatible with the SIM and must be able to function in the country you want to use
    \item
  \end{itemize}

  \subsubsection{Paid or prepaid Hostspot for use} \label{ch:prepaid-router}

  \paragraph*{Service Offer}
  Restaurants, hotels, airports, train / bus stations.

  \paragraph*{Advantages}
  \begin{itemize}
    \item It can be rented by time slots.
  \end{itemize}

  \paragraph*{Disadvantages}
  \begin{itemize}
    \item It can be more expensive if it is used intensively than other options.
  \end{itemize}


  \subsubsection{Data prepaid SIM} \label{ch:prepaid-sim}
  \paragraph*{Service Offer}

  SIM card providers valid in different countries or with specific suppliers for specific countries

  Examples :
  \begin{itemize}
    \item \href{https://www.gmyle.com/}{GMYLE}
    \item \href{https://www.americantravelsimcard.com/}{americantravelsimcard}
  \end{itemize}

  \paragraph*{Advantages}

  \begin{itemize}
    \item They can be used on any mobile device that allows the use of a SIM card and is compatible with the one provided by the provider. Type of devices: mobile, tablet, router / hotspot, laptop that accepts the SIM card.
    \item It can be used anywhere where there is mobile coverage compatible with the SIM.
  \end{itemize}


  \paragraph*{Disadvantages}

  The user must take into account:

  \begin{itemize}
    \item SIM compatibility with the user device.
    \item That the SIM and itsthe user device can operate in the visited country.
  \end{itemize}

  \chapter{Proof Of Concept (POC)} \label{sec:poc}

  \section{Summary}\label{sec:poc-summary}

  \textbf{ETH-Paid internet access} brings together:

  \begin{itemize}
    \item Ethereum holders willing to take advantage of their balances in daily life, e.g. in order to gain premium internet access.
    \item Organizations that are interested in offering internet access points
  to ethereum holders.
  \end{itemize}



  Advantages of the use of crypto assets as payment method are intended to be:
  \begin{itemize}
    \item ETH-Paid hotspot provider targets the growing ETH holders community  as customers.
    \item Allows foreigners in transit to enjoy Internet Access while saving them the costs and the complexity linked to cross currency payments.
    \item Avoids Internet Service Providers the complex management of roaming.
    \item Avoid users of the concerns linked to the understanding of the roaming rules currently in place.
  \end{itemize}

  Advantages of the use of an Ethereum smart contract are intended to be:
  \begin{itemize}
    \item Service therms and conditions are clear and publicly accessible.
    \item The provider hasn’t got the chance of alter or distort the rules.
    \item Due to the stake schema that the contract puts in place, user knows for sure that provider will do his best in order to reach agreed service level, and, when not possible, will free the funds or tokens that user previously paid.
  \end{itemize}

	\section{Business Plan}\label{sec:business-plan}
	\textbf{REVISAR esta sección}\\
  Prepaid for immediate use with time limitation. Once the payment is confirmed, the usage stopwatch starts running decreasingly.

  Payment with the ethereum cryptocurrency is allowed. Own tokens are issued to be paid with Ethers. To encourage your purchase and use, better offers will be offered than if you pay directly with Ethers. These offers can be:

  \begin{itemize}
    \item Cheaper time slots.
    \item Time slots that can only be used if tokens are owned. Example: 24 hours, 1 week, etc.
  \end{itemize}

  Payment by End User :

  \begin{itemize}
    \item Free 5 minutes test(No guarantee) : A user can only try it once a day. To detect the user, the device's Mac is used. It is stored in a Macs table for 24 hours.If a user has used it for the past 24 hours, a message appears that you cannot use the free service until after 24 hours and you are given the option of using one of the services of payment with crypto / Token. Disadvantages :
    \begin{itemize}
      \item The user changues de device.
      \item Mac Spoofing : A software can changue the Mac address.
    \end{itemize}
    \item CryptoMoney payment: Only with Ethereum. Valid for : 1 hour, 2 hours
    \item Token payment: Our own Token called IntacTok. Valid for : 1 hour, 2 hour, 24 hour.
  \end{itemize}

	\subsection{Payment model: implications}

  Several payment models would fit the aim of this POC, namely, to allow the payment through some crypto asset.

  Let's see three of them detailed bellow.


	\subsubsection{NO Smart Contract}

  The simplest option. The captive portal performs a direct payment to a private Ethereum address owned by the hotspot handler.

	\subsubsection{ETH payment under a Smart Contract}

  Thanks to the Smart Contract, the hotspot handler is able to put in place the stake schema. The use of the stake schema proves the engagement of the handler, and gives the user the control over the funds transfer termination.

  \subsubsection{Token payment under a Smart Contract}

  Purchase of dedicated tokens is publicly available. These tokens will be used to pay for Wifi connections at lower prices than in ETH payments. Token option is intended for frequent users looking for longer connections. Therefor, in addition to the stake schema, the user benefits from lower prices in exchange of his loyalti.

	\subsection{Legislation} \label{ch:legislation}
  \textbf{TODO}

  \section{Design} \label{ch:design}
  \textbf{TODO}

  \subsection{Roles} \label{ch:roles}
  \subsubsection{ETH-Paid hotspot provider (E-Php)}

  Organization willing to offer a hotspot (WiFi) service.

  Accepted payment methods will be both Ethereum (ETH) and Internet Access Token (IAT).

  Service will be provided in areas owned by the Organization, or areas where the Organization has been granted permission to offer this service.

  Potentially interested organizations:

  \begin{itemize}
    \item City councils
    \item Financial district councils
    \item Trade fairs
    \item Airport lounges
    \item Co-workings
    \item Campuses
    \item Cruise lines
    \item Malls
    \item Tourist Info Centers
  \end{itemize}

  \subsubsection{hotspot handler (hh)}

  Smart-Contract owner.

  In charge of the hotspot installation and configuration.

  \textbf{hh} holds an agreement with \textbf{E-Php} setting up their business
  relationship. It would be presumably a conventional contract
  (off-chain), which main goal is to set the compensation \textbf{E-Php} will
  receive in exchange of delivering to \textbf{hh}:

  \begin{enumerate}
    \item permission and location to perform his activity,
    \item internet access
  \end{enumerate}

  \textbf{Nota Lluis G 02/07/2019: en una solución más ambiciosa, la relación entre E-Php y hh también se podría meter en el smart-contract.}

  The hotspot is implemented over a Raspberry Pi.

  Raspberry Pi serves both WiFi signal and a captive portal that is in practice a WebApp in charge of the Ethereum (or Token) payment.

  Raspberry Pi device will ensure that connected devices that haven’t paid yet, have restricted internet access, addressed exclusively to the payment related communications, -those handled by the captive portal.

  \subsubsection{Upstream Internet Service Provider (U-ISP)}

  Access provider ISP that provide Internet access, employing a range of technologies to connect users to their network (e.g. Movistar, Vodafone,  etc.).

  \textbf{E-Php} is due to hold an agreement with the \textbf{U-ISP}, what in fact turns
  \textbf{E-Php} into a Transit ISP. Regulation could obligate \textbf{E-Php} to be formally registered as ISP.

  \subsubsection{User}

  Individual owning an ETH account with some balance. Occasionally stops himself for a while in an area where \textbf{ETH-Paid internet access} is available.

  \subsection{Roles Schemes} \label{ch:roles-schemes}

  \begin{enumerate}
    \item \textbf{E-Php}, \textbf{hh} and \textbf{U-ISP} roles are played by an unique organization.
    \item \textbf{E-Php}, \textbf{hh} and \textbf{U-ISP} roles are played by separate organizations.
    \item \textbf{E-Php} and \textbf{hh} roles are played by an unique organization \textbf{A}, and \textbf{U-ISP} is a separate organization \textbf{B}
    \item \textbf{U-ISP} and \textbf{E-Php} roles are played by an unique organization \textbf{A}, and \textbf{hh} is a separate organization \textbf{B}
    \item \textbf{U-ISP} and \textbf{hh} roles are played by an unique organization \textbf{A}, and \textbf{E-Php} is a separate organization \textbf{B}
  \end{enumerate}

  \subsection{Selected scheme}
  \textbf{REVIEW}
  Authors have chosen to develop scheme number 3 (\textbf{E-Php} and \textbf{hh} roles are played by an unique organization \textbf{A}, and \textbf{U-ISP} is a separate organization \textbf{B}) in this PoC.

  \subsection{Hotspot over a Raspberry Pi}

  Raspberry Pi is the device that will perform the hotspot function.

  \subsection{Reliance on MetaMask}
  \subsubsection{Laptop devices}
  Due to the reliance on MetaMask assumed in this PoC, only devices supporting this web extension will be suitable to perform any test.

  \subsubsection{Mobile devices}

  Currently Metamask Mobile is still under development what therefor prevents Smartphones from being used in this PoC.

  \begin{itemize}
    \item \href{https://medium.com/metamask/metamask-monthly-june-2dddbb6618a3}{metamask-monthly-june}
    \item \href{http://mobile.metamask.io/}{mobile.metamask.io}
  \end{itemize}

  \section{How does it work?}
  \subsection{Installation}

  \subsubsection{Hotspot over a Raspberry Pi}

  \textbf{E-Php} installs the hotspot over a Raspberry Pi.

  \begin{itemize}
    \item Is connected to a router and has full internet access according to the agreement with the \textbf{U-ISP}.
    \item Is running \textbf{etherwall}: Back-end to control network ip packet forwarding in order to allow ISP customers to share their internet connection using WiFi depending on end users performing token/ethereum payment transactions.
    \item Serves open WiFi labeled as \textbf{ETH Internet Access}.
  \end{itemize}

  \subsubsection{Smart contract deployment}

  \textbf{E-Php} deploys the smart contract in Ethereum.

  Through deployment, \textbf{E-Php} is due to set the ethereum account which will be the ETH source for stake purpose.

  \subsection{Start of service through payment}


  \subsubsection{User payment}

  \textbf{User} connects to \textbf{ETH Internet Access} open WiFi and obtains a redirect to the captive portal frontend.

  \textbf{User} clicks one of the available \textbf{Buy Now ...} buttons in order to purchase a temporary full internet access.

  The WebApp uses MetaMask to allow \textbf{User} to sign a payment transaction of ETH or IA Tokens.

  \subsubsection{\textbf{E-Php} stake}

  Payment transaction

  Once the transaction is confirmed (or just submitted, whenever WebApp could manage a courtesy lapse), WebAPP backend hosted by the Raspberry Pi) sets up rules needed over iptables to forward and retrieve user's traffic based on device's MAC address.

  At this point, service has started and \textbf{User} enjoys full internet access.

  \subsection{End of service}

  Backend keeps internal track of each connected device MAC address and periodically checks end of service condition; in case the user has no service time left, it revokes connectivity deleting the associated MAC address rules over iptables.

  \subsection{While on service}
  \textbf{TODO}

  \subsubsection{What about showing \textbf{User} his remaining service time?}
  \textbf{TODO}
  \subsubsection{What about letting \textbf{User} know his looming service termination?}
  \textbf{TODO}
  \subsubsection{What about reconnecting after an unintentional break?}
  \textbf{TODO}


  \subsection{Architecture} \label{ch:architecture}
  \textbf{TODO}

  \section{Implementation} \label{ch:implementation}
  \textbf{TODO}

  \subsection{Front-end} \label{ch:front-end}
  \textbf{ See yellowPaper}

  \subsection{Back-end} \label{ch:back-end}
  \textbf{ See yellowPaper}

	\subsubsection{Smart contracts} \label{ch:smart-contracts}
  \textbf{See yellowPaper}


  \subsubsection{Raspberry Pi} \label{ch:raspberry}
  \textbf{See yellowPaper}

  \subsection{Tests} \label{ch:test}
	\textbf{See yellowPaper}

  \subsubsection{Validation Tests} \label{sec:validation-test}
  \textbf{See yellowPaper}

  \subsubsection{Deployment} \label{sec:deployment}
  \textbf{See yellowPaper}


  \chapter{Conclusions} \label{sec:conclusions}
  \textbf{TODO}

  \chapter{Future Work} \label{sec:future-work}
	\textbf{See yellowPaper}

	\section{Business} \label{ch:furute-business}
  \textbf{TODO}


	\section{Technical} \label{ch:future-technical}
  \textbf{TODO}

  \subsection{DAI Payment} \label{ch:dai-payment}
  \textbf{See yellowPaper}

  \subsection{Other Hardware} \label{ch:other-hardware}
  \subsection{Raspberry Pi 4 Modelo B} \label{ch:raspberry4}
  \textbf{See yellowPaper}

  \subsection{Docker Swarm} \label{ch:docker-swarm}
  \textbf{See yellowPaper}

  \chapter*{References}
  \textbf{See yellowPaper}

\end{document}
